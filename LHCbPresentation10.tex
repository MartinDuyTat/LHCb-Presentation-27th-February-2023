%%%%%%%%%%%%%%%%%%%%%%%%%%%%%%%%%%%%%%%%%%%%%%%%%%%%%%%%%%%%%%%%%%%%%%
% Overleaf (WriteLaTeX) Example: Molecular Chemistry Presentation
%
% Source: http://www.overleaf.com
%
% In these slides we show how Overleaf can be used with standard 
% chemistry packages to easily create professional presentations.
% 
% Feel free to distribute this example, but please keep the referral
% to overleaf.com
% 
%%%%%%%%%%%%%%%%%%%%%%%%%%%%%%%%%%%%%%%%%%%%%%%%%%%%%%%%%%%%%%%%%%%%%%

\documentclass{beamer}

\mode<presentation>
{
  \usetheme{Madrid}       % or try default, Darmstadt, Warsaw, ...
  \usecolortheme{default} % or try albatross, beaver, crane, ...
  \usefonttheme{default}    % or try default, structurebold, ...
  \setbeamertemplate{navigation symbols}{}
  \setbeamertemplate{caption}[numbered]
} 

\usepackage[english]{babel}
\usepackage[utf8x]{inputenc}
\usepackage{graphicx}
\usepackage{hyperref}
  \hypersetup{colorlinks=true}
  \hypersetup{urlcolor=blue}
  \hypersetup{linkcolor = .}
\usepackage{xcolor}
\usepackage{siunitx}
  \sisetup{separate-uncertainty = true}
\usepackage{physics}
\usepackage[font=small,labelfont=bf,justification=centering]{caption}
\usepackage{subcaption}
\usepackage[en-GB]{datetime2}
\usepackage{overpic}
\usepackage{feynmp}
\DeclareGraphicsRule{*}{mps}{*}{}
\usepackage{scalerel}
\newcommand{\mylbrace}[2]{\vspace{#2pt}\hspace{6pt}\scaleleftright[\dimexpr5pt+#1\dimexpr0.06pt]{\lbrace}{\rule[\dimexpr2pt-#1\dimexpr0.5pt]{-4pt}{#1pt}}{.}}
\newcommand{\myrbrace}[2]{\vspace{#2pt}\scaleleftright[\dimexpr5pt+#1\dimexpr0.06pt]{.}{\rule[\dimexpr2pt-#1\dimexpr0.5pt]{-4pt}{#1pt}}{\rbrace}\hspace{6pt}}
\usepackage{ulem} % Line across text

% Here's where the presentation starts, with the info for the title slide
\title[$K^+K^-\pi^+\pi^-$]{\texorpdfstring{$D\to K^+K^-\pi^+\pi^-$}{K+K-pi+pi-} strong phase analysis at BESIII}

\author{Martin Tat}
\institute{Oxford LHCb}
\date{27th February 2023}

\titlegraphic{\includegraphics[height = 2cm]{lhcb.jpg}\hspace{1cm}~%
              \includegraphics[height = 2cm]{OxfordLogo.pdf}\hspace{1cm}~%
              \includegraphics[height = 2cm]{bes3.jpg}}

\begin{document}

\begin{frame}
  \titlepage
\end{frame}

% These three lines create an automatically generated table of contents.
%\begin{frame}{Outline}
%  \tableofcontents
%\end{frame}

\section{Recap of BESIII analysis}

\begin{frame}{Recap of BESIII analysis}
  \begin{center}
    \Large{Analysis of $D^0\to K^+K^-\pi^+\pi^-$}
  \end{center}
  \vspace{0.5cm}
  \begin{itemize}
    \setlength\itemsep{1.0em}
    \item{Study $D^0$-$\bar{D^0}$ strong phase difference in bins of the 5D phase space}
    \item{Measurement of amplitude averaged strong phases $c_i$ and $s_i$}
    \item{$c_i$ and $s_i$ are important inputs to the $\gamma$ measurement at LHCb}
    \begin{itemize}
      \item{LHCb result: $\gamma = (116^{+12}_{-14})^\circ$ with model dependent inputs}
      \item{$\gamma$ may change when updated with model independent $c_i$ and $s_i$}
    \end{itemize}
    \item{Measurement technique unique to charm factories: Study decays of quantum correlated $D\bar{D}$ pairs using a double tag method}
  \end{itemize}
\end{frame}

\begin{frame}{Recap of BESIII analysis}
  \begin{itemize}
    \item{$\psi(3770)\to D^0\bar{D^0}$ decay conserves $\mathcal{C} = -1$}
  \end{itemize}
  \begin{figure}[H]
    \centering
    \vspace{-1.5cm}
    \begin{fmffile}{fgraph/fgraph_ee1}
      \setlength{\unitlength}{1cm}
      \begin{fmfgraph*}(8,5)
        \fmfleft{i}
        \fmfright{o}
        \fmflabel{$D^0$}{i}
        \fmflabel{$\bar{D^0}$}{o}
        \fmf{fermion}{w,i}
        \fmf{fermion}{w,o}
        \fmfblob{1cm}{w}
        \fmfv{label=$\psi(3770)$,label.dist=15,label.angle=90}{w}
      \end{fmfgraph*}
    \end{fmffile}
    \vspace{-2.0cm}
  \end{figure}
  \begin{itemize}
    \item{But since they are quantum correlated, we must consider their CP eigenstates $D_\pm = (\lvert D^0\rangle\pm\lvert\bar{D^0}\rangle)/\sqrt{2}$}
    \item{Total wavefunction is $\lvert D^0\rangle\lvert\bar{D^0}\rangle - \lvert\bar{D^0}\rangle\lvert D^0\rangle = \lvert D_+\rangle\lvert D_-\rangle + \lvert D_-\rangle\lvert D_+\rangle$}
  \end{itemize}
  \begin{figure}[H]
    \centering
    \vspace{-1.5cm}
    \begin{fmffile}{fgraph/fgraph_ee2}
      \setlength{\unitlength}{1cm}
      \begin{fmfgraph*}(8,5)
        \fmfleft{i}
        \fmfright{o}
        \fmflabel{$D_+$}{i}
        \fmflabel{$D_-$}{o}
        \fmf{fermion}{w,i}
        \fmf{fermion}{w,o}
        \fmfblob{1cm}{w}
        \fmfv{label=$\psi(3770)$,label.dist=15,label.angle=90}{w}
      \end{fmfgraph*}
    \end{fmffile}
    \vspace{-2.0cm}
  \end{figure}
  \begin{center}
    The two $D$ mesons do \underline{not} communicate, but the $D\to KK\pi\pi$ decay is perfectly correlated with the tagged $D$
  \end{center}
\end{frame}

\begin{frame}{Strong-phases in quantum correlated $D^0\bar{D^0}$ decays}
  \begin{itemize}
    \item{Tag mode can be a \underline{CP even tag}}
    \begin{itemize}
      \item{$KK$, $\pi\pi$, $\pi\pi\pi^0$, $K_S\pi^0\pi^0$, $K_L\pi^0$, $K_L\omega$}
    \end{itemize}
  \end{itemize}
  \begin{figure}[H]
    \centering
    \vspace{0.3cm}
    \begin{fmffile}{fgraph/fgraph_CPeven_tag}
      \setlength{\unitlength}{1cm}
      \begin{fmfgraph*}(8,4)
        \fmfstraight
        \fmfleft{i4,i3,i2,i1}
        \fmfright{g1,o1,o2,g2}
        \fmflabel{$K^+$}{o1}
        \fmflabel{$K^-$}{o2}
        \fmflabel{$K^+$}{i1}
        \fmflabel{$K^-$}{i2}
        \fmflabel{$\pi^+$}{i3}
        \fmflabel{$\pi^-$}{i4}
        \fmf{fermion}{w,i1}
        \fmf{fermion}{w,i2}
        \fmf{fermion}{w,i3}
        \fmf{fermion}{w,i4}
        \fmf{fermion}{w,o1}
        \fmf{fermion}{w,o2}
        \fmf{phantom}{w,g1}
        \fmf{phantom}{w,g2}
        \fmfblob{1cm}{w}
      \end{fmfgraph*}
    \end{fmffile}
    \vspace{0.3cm}
  \end{figure}
  \begin{center}
    $D\to K^+K^-$, which is $C\!P$ even, forces $D\to K^+K^-\pi^+\pi^-$ to be $C\!P$ odd
  \end{center}
\end{frame}

\begin{frame}{Strong-phase in quantum correlated $D^0\bar{D^0}$ decays}
  \begin{itemize}
    \item{Tag mode can be a \underline{CP odd tag}}
    \begin{itemize}
      \item{$K_S\pi^0$, $K_S\omega$, $K_S\eta$, $K_S\eta'$, $K_L\pi^0\pi^0$}
    \end{itemize}
  \end{itemize}
  \begin{figure}[H]
    \centering
    \vspace{0.3cm}
    \begin{fmffile}{fgraph/fgraph_CPodd_tag}
      \setlength{\unitlength}{1cm}
      \begin{fmfgraph*}(8,4)
        \fmfstraight
        \fmfleft{i4,i3,i2,i1}
        \fmfright{g1,o1,o2,g2}
        \fmflabel{$\pi^0$}{o1}
        \fmflabel{$K_S$}{o2}
        \fmflabel{$K^+$}{i1}
        \fmflabel{$K^-$}{i2}
        \fmflabel{$\pi^+$}{i3}
        \fmflabel{$\pi^-$}{i4}
        \fmf{fermion}{w,i1}
        \fmf{fermion}{w,i2}
        \fmf{fermion}{w,i3}
        \fmf{fermion}{w,i4}
        \fmf{fermion}{w,o1}
        \fmf{fermion}{w,o2}
        \fmf{phantom}{w,g1}
        \fmf{phantom}{w,g2}
        \fmfblob{1cm}{w}
      \end{fmfgraph*}
    \end{fmffile}
    \vspace{0.3cm}
  \end{figure}
  \begin{center}
    $D\to K_S^0\pi^0$, which is $C\!P$ odd, forces $D\to K^+K^-\pi^+\pi^-$ to be $C\!P$ even
  \end{center}
\end{frame}

\begin{frame}{Fit yield of some CP tags}
  \begin{center}
    Do simultaneous double tag yield fit of CP tags
  \end{center}
  \begin{figure}
    \centering
    \begin{subfigure}{0.49\textwidth}
      \includegraphics[width = 1.0\textwidth, trim = {0 14cm 0 0}, clip = true]{example-image-a}
      \caption{$10.2^{+6.7}_{-3.9}$}
    \end{subfigure}%
    \begin{subfigure}{0.49\textwidth}
      \includegraphics[width = 1.0\textwidth, trim = {0 14cm 0 0}, clip = true]{example-image-a}
      \caption{$14.4^{+4.8}_{-4.1}$}
    \end{subfigure}
    \caption{$KK\pi\pi$ vs $KK$}
  \end{figure}
\end{frame}

\begin{frame}{Summary and next steps}
  \begin{itemize}
    \setlength\itemsep{0.8em}
    \item{BESIII measurement of $c_i$ and $s_i$ is progressing well}
    \item{A partially reconstructed $D\to KK\pi\pi$ method has been tested, but there were some challenges with large $D\to K\pi\pi\pi\pi^0$ backgrounds}
    \item{The preliminary fit of $c_i$ and $s_i$ shows promising results}
    \begin{itemize}
      \item{A method for direct DCS decay corrections is working well}
      \item{Results of $c_i$ agree with the $F_+$ measurement}
      \item{$s_i$ shows tensions with the LHCb model}
    \end{itemize}
    \item{Next steps:}
    \begin{enumerate}
      \item{Finish calculation of peaking backgrounds in each bin}
      \item{Reprocess all data and generate new MC once new data is available}
      \item{Add the rest of the tags}
      \item{Charm WG review}
    \end{enumerate}
  \end{itemize}
  \begin{center}
    \huge Thank you for listening!
  \end{center}
\end{frame}

\end{document}
